\documentclass[12pt]{article}
\usepackage{lingmacros}
\usepackage{tree-dvips}
\usepackage{amsmath,amsthm,amssymb}
\usepackage{mathtext}
\usepackage[T1,T2A]{fontenc}
\usepackage[utf8]{inputenc}
\usepackage[english,russian]{babel}
\usepackage{geometry}
\usepackage{amsmath}
\geometry{left=3cm}
\geometry{right=1.5cm}
\geometry{top=2cm}
\geometry{bottom=2cm}
\begin{document}

\section*{Непрерывные математические модели}

Модифицированным методом Ньютона-Канторовича найти первые два приближения к решению краевой задачи

\begin{gather}
    -\ddot{x} + 2t^{-10}x^3 = -10t^2 \label{eq:task}
\end{gather}
\begin{gather}
    x(1)=1,\ x(2)=16 \label{eq:limit}
\end{gather}

Точное решение: $x^*(t) = t^4$

\subsection*{Расчетные формулы}



\begin{gather}
\ddot{x}(t) + f(t,\ x_0(t),\ \dot{x}_0(t)) = 0
\end{gather}
\begin{multline}
	-\ddot{x}_{n+1}(t) + f'_x(t,\ x_0(t),\ \dot{x_0}(t))\ x_{n+1}(t) +
	f'_{\dot{x}}(t,\ x_0(t),\ \dot{x}_0(t))\ \dot{x}_{n+1}(t) = \\
	f'_x(t,\ x_0(t),\ \dot{x_0}(t))\ x_n(t) +
	f'_{\dot{x}}(t,\ x_0(t),\ \dot{x}_0(t))\ \dot{x}_n(t) -
	f(t,\ x_n(t),\ \dot{x}_n(t))
\end{multline}
\begin{equation}
	x_{n+1}(a) = \alpha_0,\ x_{n+1}(b) = \alpha_1
\end{equation}

Для задачи \eqref{eq:task} - \eqref{eq:limit} имеем:
\begin{gather*}
	-\ddot{x} + 2t^{-10}x^3 + 10t^2 = 0 \\
	f(t,\ x(t),\ \dot{x}(t)) = 2t^{-10}x^3(t) + 10t^2 \\
	f(t,\ x_n(t),\ \dot{x}_n(t)) = 2t^{-10}x^3_n(t) + 10t^2 \\
	f'_x(t,\ x_0(t),\ \dot{x}_0(t)) = 6t^{-10}x^2_0(t) \\
	f'_{\dot{x}}(t,\ x_0(t),\ \dot{x}_0(t)) = 0
\end{gather*}

\begin{equation}
    -\ddot{x}_{n+1}(t) + 6t^{-10}x^2_0(t)x_{n+1}(t) =
     6t^{-10}x^2_0(t)x_n(t) - 2t^{-10}x^3_n(t) - 10t^2
\end{equation}
\begin{equation}
    x_{n+1}(1)=1,\ x_{n+1}(2)=16
\end{equation}

Введем замену $x_{n+1}(t) = y_{n+1}(t) + G(t)$, где $G(t)$ имеет вид:
\begin{gather*}
	G(t) = \alpha \frac{b-t}{b-a} + \beta \frac{t-a}{b-a} \\
	G(t) = 1 (2-t) + 16 (t-1) \\
	G(t) = 2-t + 16t-16 \\
\end{gather*}
\begin{equation}
	G(t) = 15t - 14
\end{equation}

\begin{multline} \label{eq:replace_task}
    -\ddot{y}_{n+1}(t) + 6t^{-10}(y_0(t) + G(t))^2\ (y_{n+1}(t) + G(t)) =
     6t^{-10}(y_0(t) + G(t))^2\ (y_n(t) + G(t)) \\
     - 2t^{-10}(y_n(t) + G(t))^3 - 10t^2
\end{multline}
\begin{equation} \label{eq:replace_limits}
    y_{n+1}(1)=0,\ y_{n+1}(2)=0
\end{equation}

\newcommand{\gone}{\ensuremath{
    6t^{-10}(y_0(t)+G(t))^2\ y_n(t) - 2t^{-10}(y_n(t)+G(t))^3 - 10t^2
}}

Приведем \eqref{eq:replace_task} - \eqref{eq:replace_limits} к следующему виду:
\begin{gather}
    p(t)\ddot{y}_{n+1}(t) + q(t)\dot{y}_{n+1}(t) + r(t)y_{n+1}(t) =
    g_1(t, y_n(t), \dot{y}_n(t)) \\
    y_{n+1}(1) = 0,\ y_{n+1}(2) = 0
\end{gather}

\begin{multline}
    -\ddot{y}_{n+1}(t) + 6t^{-10}(y_0(t) + G(t))^2\ y_{n+1}(t) = \gone
\end{multline}
\begin{equation}
    y_{n+1}(1) = 0,\ y_{n+1}(2) = 0
\end{equation}

Первое приближение:
\begin{equation}
    y_1^N(t) = \sum_{k = 1}^{N}a_ke_k(t)
\end{equation}

Метод наименьших квадратов:

\newcommand{\operator}[1]{\ensuremath{
    -\ddot{e}_#1(t) + 6t^{-10}(y_0(t) + G(t))^2\ e_#1(t)
}}

\begin{multline} \label{lsm}
    \sum_{k = 1}^{N}a_k
    \int_1^2 [\operator{k}][\operator{j}]dt = \\
    \int_1^2 [\gone] \\ [\operator{j}]dt,\ \ \
    j = \overline{1,N}
\end{multline}

В качестве координатной системы используются тригонометрические многочлены:
\begin{equation}
    \begin{split}
        e_k(t) &= \sin{\pi k (t - 1)} \\
        \ddot{e}_k(t)&= -\pi^2 k^2\sin{\pi k(t-1)}
    \end{split}
\end{equation}
\end{document}
